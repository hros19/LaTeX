\documentclass[conference]{IEEEtran}
\IEEEoverridecommandlockouts
% The preceding line is only needed to identify funding in the first footnote. If that is unneeded, please comment it out.
\usepackage{cite}
\usepackage{amsmath,amssymb,amsfonts}
\usepackage{algorithmic}
\usepackage{graphicx}
\usepackage{textcomp}
\usepackage{xcolor}
\usepackage[spanish]{babel}

\renewcommand{\IEEEkeywordsname}{Palabras clave}

\def\BibTeX{{\rm B\kern-.05em{\sc i\kern-.025em b}\kern-.08em
    T\kern-.1667em\lower.7ex\hbox{E}\kern-.125emX}}
\begin{document}

\title{Survey de análisis de sentimientos, minería de argumentos y opiniones\\
}

\author{\IEEEauthorblockN{
  Hansol Antay Rostrán}
\IEEEauthorblockA{\textit{Tecnológico de Costa Rica} \\
\textit{Escuela de Computación}\\
Limón, Costa Rica \\
rostrhan@outlook.com}
\and
\IEEEauthorblockN{Fredd Badilla Víquez}
\IEEEauthorblockA{\textit{Tecnológico de Costa Rica} \\
\textit{Escuela de Computación}\\
Limón, Costa Rica \\
frbadilla@estudiantec.cr}
\and
\IEEEauthorblockN{Randall Corella Castillo}
\IEEEauthorblockA{\textit{Tecnológico de Costa Rica} \\
\textit{Escuela de Computación}\\
Limón, Costa Rica  \\
randallcorella@estudiantec.cr}
}

\maketitle

\begin{abstract}
  En este survey se va a dar un repaso sobre el estado actual de los análisis de sentimientos en la era digital moderna. 
  Se denotarán conceptos básicos, técnicas, métodos, acercamientos, desafíos o limitaciones, junto con aplicaciones que se utilizan hoy en día. 
  Se analizarán desde modelos básicos para procesamiento de lenguaje natural, hasta modelos que utilizan aprendizaje automático. 
  La idea es brindar distintos aspectos que conllevan el análisis de sentimientos para compararlos y dar a conocer cómo es que se realizan estos procesos. 
  Se estudiarán las aplicaciones como el análisis de retroalimentación brindado por clientes de productos y servicios, el análisis de reputación de una empresa u organización, la predicción de movimientos sociales con potencial negativo, el estudio de la opinión pública sobre las campañas gubernamentales, el modelado de preferencias de usuario para anuncios, recomendaciones y demás. 
  Se pretende concluir este survey con un repaso de los conceptos más importantes, señalar los métodos y procedimientos para utilizados para el análisis de sentimientos, y el impacto que tiene este en la era digital tanto para las empresas como para los propios individuos.
\end{abstract}

\begin{IEEEkeywords}
component, formatting, style, styling, insert
\end{IEEEkeywords}

\section{Introducción}
El análisis de sentimientos es una disciplina que no es nueva, sino que se ha consolidado como una rama del procesamiento del lenguaje natural desde la década de 1950. 
En los últimos años, el análisis de sentimientos ha adquirido una gran importancia debido al auge de las redes sociales y la creciente disponibilidad de tecnologías para el procesamiento y modelado de la semántica del lenguaje. 
Para comprender su relevancia, es necesario plantearse cómo las plataformas digitales son capaces de recomendar canciones, vídeos o productos y cómo un simple tweet, comentario de Facebook o reseña en Amazon puede afectar significativamente a una empresa u organización. 
En este estudio, profundizaremos en los procesos utilizados por empresas u organizaciones en la toma de decisiones a partir de la comprensión de los conceptos básicos del análisis de sentimientos y el procesamiento del lenguaje natural.

En este trabajo, se explorarán diversos aspectos del análisis de sentimientos, desde los fundamentos computacionales hasta los aspectos psicológicos, y se describirán las técnicas y algoritmos utilizados por distintos procedimientos. 
Aunque se abordarán las limitaciones del análisis de sentimientos, como la detección de sarcasmo, ironía, negación o rechazo en el análisis textual, se evitará abordar temas ético-morales y de privacidad de los usuarios, ya que todos los métodos se basan en la recopilación masiva de datos públicos.

En una sección dedicada exclusivamente a los casos de uso, se presentarán ejemplos de aplicaciones del análisis de sentimientos en distintas industrias, enfocándose principalmente en la recopilación textual de texto, pero también mencionando otras aplicaciones que utilizan diferentes tecnologías. 
A través de estos casos prácticos, se ilustrarán los usos puntuales de dichos métodos en la vida cotidiana.

En la conclusión de este estudio, se repasarán las técnicas y aplicaciones del análisis de sentimientos que se han mencionado, comparado y estudiado. 
Se destacará la importancia e impacto que ha tenido el análisis de sentimientos en la industria, los hitos de las investigaciones mencionadas en todo el documento, la evolución de esta área en el ámbito computacional y empresarial, las limitaciones del campo de estudio y el alcance de futuras investigaciones relacionadas con el análisis de sentimientos.

\section{Conceptos fundamentales}
La presente sección es una reflexión sobre la definición de sentimientos, emociones y pensamiento, así como sobre la viabilidad del análisis de sentimientos desde una perspectiva computacional.

A pesar de que el enfoque principal sea computacional, todo va a girar en torno a los sentimientos, lo que lleva a la pregunta de cómo definir algo tan subjetivo. 
En este sentido, se debe hacer hincapié en que los sentimientos son el principal objeto de estudio y que para entenderlos es necesario conocer primero el ámbito psicológico y las emociones. 
Se define la emoción como un conjunto de reacciones psicológicas y fisiológicas generadas por el sistema límbico del cerebro y causadas por estímulos internos o externos que nos predisponen a reaccionar de cierta manera ante la situación. 
En cambio, los sentimientos son estados prolongados que se originan de la conceptualización racional de las emociones.

Una vez definidos estos conceptos, se plantea la cuestión de la viabilidad del análisis de sentimientos desde una perspectiva computacional. 
Se reconoce que el análisis de sentimientos es un problema abstracto, complejo y subjetivo, pero que, gracias a los avances tecnológicos, se han encontrado formas de captar, procesar e identificar las emociones y sentimientos expresados por los usuarios en diferentes plataformas. 

Detallaremos a continuación los conceptos básicos del análisis de sentimientos, es importante comprenderlos para poder adentrarnos más en casos de uso, aplicaciones y algoritmos o técnicas que son utilizadas en la actualidad. Todo esto con el fin de comparar y denotar los puntos fuertes y débiles de cada uno de ellos a medida que vamos presentando los casos de uso.

\subsection{Transfondo psicológico}
Tal como aplican en el estudio \cite{b1} del análisis de sentimientos en redes sociales, se debe tener en cuenta que este es un campo de estudio que se encuentra en constante evolución, por lo que es necesario tener en cuenta los avances que se han ido produciendo en el campo de la psicología. En dicho survey mencionan tres componentes principales de las emociones: la experiencia subjetiva, la respuesta fisiológica y la expresión conductual. Y su vez, lo diferencian del sentimiento, como la actitud mental que se desarrolla a partir de la experiencia emocional.

Incluso quitando un ámbito computacional al tema de análisis de sentimientos, \cite{b2} concluye que las emociones son una parte fundamental de la vida humana, ya que afectan nuestras decisiones y nos permiten comunicarnos mejor con el mundo. 
La detección de emociones implica identificar los diferentes sentimientos o emociones de una persona, pero detectar emociones a partir de texto es complicado debido a la gran cantidad de ambigüedades y jerga que se usan a diario. 
Además, la detección de emociones no se limita a identificar solo las condiciones psicológicas primarias (felicidad, tristeza, enojo), sino que puede llegar hasta 6 o 8 escalas según el modelo de emoción utilizado.

Como se mencionó anteriormente, pueden existir diferentes modelos de emoción, pero todos ellos se basan en la misma idea: identificar las emociones a partir de la expresión de un texto. Vamos a señalar dos categorías de modelos que servirán como ápice de muchas metodologías aplicadas en el análisis de sentimientos. En \cite{b3}, se menciona primero el modelo categórico de emociones. Este clasifica las emociones en categorías discretas y específicas, como la alegría, la tristeza, el miedo o la ira. Este modelo ha sido ampliamente utilizado en la psicología y en la investigación de las emociones.

Por otro lado, el modelo dimensional de emociones define las emociones como puntos en un espacio multidimensional, donde cada dimensión representa un aspecto emocional específico. Las dimensiones más comunes incluyen el valencia (positiva o negativa), la activación (alta o baja) y el control (alto o bajo).

A diferencia del modelo categórico, el modelo dimensional sugiere que las emociones no son mutuamente exclusivas y pueden existir en un continuo o espectro. Este modelo ha sido utilizado en la investigación de la neurociencia afectiva y en la psicología clínica para ayudar a comprender las emociones en un contexto más amplio.

\subsection{Limitaciones computacionales}
A pesar de los avances en el análisis de sentimientos desde una perspectiva computacional, todavía hay ciertas limitaciones en cuanto a la comprensión de los sentimientos y emociones expresados por los usuarios. 
Una de las limitaciones más importantes es la subjetividad inherente al análisis de sentimientos. 
Aunque los algoritmos pueden ser entrenados para reconocer ciertos patrones y palabras asociadas con emociones específicas, aún hay muchas expresiones que pueden ser interpretadas de diferentes maneras según el contexto o la cultura. 
Por lo tanto, el análisis de sentimientos no es una tarea fácil y requiere la implementación de técnicas avanzadas de procesamiento del lenguaje natural y aprendizaje automático.

Además, otro desafío importante es el procesamiento de grandes volúmenes de datos. 
El análisis de sentimientos implica el procesamiento de grandes cantidades de texto, lo que puede resultar en una sobrecarga de información. 
Los sistemas de análisis de sentimientos deben ser capaces de manejar grandes cantidades de datos de manera eficiente y efectiva, lo que requiere una gran cantidad de poder de procesamiento y almacenamiento de datos.

Otra limitación importante del análisis de sentimientos es la falta de contexto. 
A menudo, las expresiones de sentimientos se dan en un contexto específico, lo que puede afectar la forma en que se interpretan. 
Por lo tanto, la falta de contexto puede llevar a una interpretación errónea de los sentimientos expresados en un texto. 
Para superar esta limitación, se han desarrollado técnicas avanzadas que utilizan el contexto para comprender mejor el significado de las expresiones de sentimientos. Además, el análisis de sentimientos también puede ser limitado por la calidad y cantidad de los datos disponibles. 
Para entrenar a los algoritmos de análisis de sentimientos, se necesitan grandes cantidades de datos de alta calidad. 
Si los datos utilizados para el entrenamiento son incompletos o no representan la realidad, los algoritmos de análisis de sentimientos pueden no ser precisos. 
Por lo tanto, es importante asegurarse de que los datos utilizados para el entrenamiento sean precisos y representativos.

A pesar de que se han realizado grandes avances en el análisis de sentimientos desde una perspectiva computacional, todavía hay varias limitaciones que deben ser superadas. 
La subjetividad inherente al análisis de sentimientos, la falta de contexto, la sobrecarga de información, la calidad y cantidad de los datos disponibles son algunas de las limitaciones más importantes que deben ser abordadas para lograr una mayor precisión en el análisis de sentimientos.

Uno de los puntos que vamos a resaltar en este survey es referente a como poder detectar sarcasmo e ironía en un texto. El trabajo de investigación \cite{b4} es todo un estudio sobre como el sarcasmo es un gran desafío para los modelos de análisis de sentimientos, ya que puede ocultar la intención real de desprecio y negatividad detrás de una representación positiva. La detección de sarcasmo es importante para comprender los verdaderos sentimientos y creencias de las personas. El sarcasmo es difícil de identificar tanto para humanos como para máquinas debido a su figuratividad y sutileza. La detección automática de sarcasmo se ha abordado como un problema de clasificación binaria y se ha centrado principalmente en recursos léxicos y pragmáticos, así como en cambios sentimentales en las oraciones. Sin embargo, el sarcasmo a menudo se manifiesta de manera implícita, lo que dificulta la identificación para las máquinas.

\subsection{Fina brecha del aprendizaje automático}
Para adentrarnos más adelante en métodos y algoritmos sofisticados que se utilizan en el análisis de sentimientos aprovecharemos la cuestión de si es necesario o no aplicar aprendizaje automático para el análisis de sentimientos. 
En \cite{b5} denotan muchos conceptos y elementos utilizados en el análisis de sentimientos. Esto nos brindará una excusa para hablar de la fina brecha que existe entre el aprendizaje automático y el análisis de sentimientos, mientras introducimos conceptos e ideas base detras de los algoritmos y métodos que se utilizan.

\subsubsection{Análisis sin aprendizaje automático}
La idea más básica que se puede dar del análisis de sentimientos es que es una recopilación de palabras y frases que se utilizan para expresar sentimientos positivos o negativos. Por lo que se puede decir que el análisis de sentimientos es una tarea de clasificación de texto. Para ello no es requerido el aprendizaje automático, se puede realizar tomando en cuenta los siguientes elementos:
\begin{itemize}
\item \textbf{Cuantificación de palabras}: El conteo, la frecuencia y la ponderación de las palabras se utilizan para cuantificar el sentimiento de un texto. Este simplemente se basa en una idea muy sencilla: \textit{contar las palabras positivas y negativas en un texto, para luego determinar si el texto es positivo o negativo}.
\item \textbf{Heurísticas}: El concepto de heurística en el ámbito de procesamiento de texto lo vamos a definir como ciertas reglas que se utilizan para determinar el sentimiento de un texto compuesto. 
A diferencia de la cuantificación de palabras, las heurísticas no se basan en el conteo de palabras, sino en el análisis de la estructura del texto. 
Por ejemplo, si un texto comienza con una palabra negativa, entonces el texto es negativo.
\item \textbf{Diccionarios de sentimientos}: Los diccionarios de sentimientos son una colección de palabras y frases que se utilizan para expresar sentimientos positivos o negativos. 
Estos diccionarios se utilizan para determinar el sentimiento de un texto. 
Por ejemplo, si un texto contiene palabras positivas, entonces el texto es positivo. 
A diferencia de las heurísticas, los diccionarios de sentimientos no se basan en la estructura del texto, sino en el contenido del texto.
\end{itemize}

\subsubsection{Análisis con aprendizaje automático}
Vamos a definir el aprendizaje automático como un conjunto de técnicas que permiten a las computadoras aprender de los datos sin ser explícitamente programadas. En este apartado vamos a mencionar elementos que nos ayudarán a entender el aprendizaje automático y su relación con el análisis de sentimientos.

\begin{itemize}
\item \textbf{Aprendizaje supervisado}: El aprendizaje supervisado es un tipo de aprendizaje automático en el que se entrena un modelo utilizando datos etiquetados. Haciendo uso de este tipo de aprendizaje, se puede entrenar un modelo para predecir el sentimiento de un texto.
\item \textbf{Representación textual}: Parte del aprendizaje supervisado es la representación textual, y va más allá de la cuantificación de palabras y las heurísticas. Depende mucho del modelo que se utilice, pero en general, la representación textual se basa en la idea de que las palabras que aparecen en un texto tienen un significado.
\item \textbf{Selección de características}: La selección de características es un paso importante en el aprendizaje supervisado. Veremos más adelante como podemos hacer uso de bolsas de palabras, n-gramas y otros elementos para seleccionar características.
\item \textbf{Hiperparametrización}: La hiperparametrización es un proceso que se utiliza para optimizar los parámetros de un modelo. Estos sirven para ajustar el modelo a los datos de entrenamiento, de manera que el modelo pueda generalizar mejor nuevos datos. Algunos hiperparámetros son: el número de capas, el número de neuronas por capa, la función de activación, el optimizador, profundidad del árbol de decisión, el número de épocas, etc.
\item \textbf{Validación cruzada}: La validación cruzada es un método de evaluación de modelos que se utiliza para evaluar la precisión de un modelo. El proceso de validación es muy sencillo, se divide el conjunto de datos en k partes, se entrena el modelo k veces, cada vez utilizando una parte diferente como conjunto de prueba y el resto como conjunto de entrenamiento. Al final se promedian los resultados de las k pruebas para obtener un resultado final.
\end{itemize}

Ya con una vez comprendidos estos elementos básicos, deberíamos tener una visión más clara sobre como funcionan procesos complejos de análisis de sentimientos, como el aprendizaje automático. De forma intuitiva, podríamos darnos una idea de como funcionan internamente la mayoría de modelos planteados para el análisis de sentimientos.

En la continuación de la sección, veremos términos más puntuales y técnicos que serán referenciados cuando expliquemos y comparemos los diferentes métodos y algoritmos que se utilizan en el análisis de sentimientos.

\subsection{Procesamiento de Lenguaje Natural}

El procesamiento de lenguaje natural (PLN) es una rama de la inteligencia artificial que se encarga de estudiar y desarrollar métodos para que las computadoras puedan procesar lenguaje natural. 
El procesamiento de lenguaje natural es un campo muy amplio, y se puede dividir en varias áreas, como la comprensión del lenguaje natural, la generación de lenguaje natural, la traducción automática, la extracción de información, la clasificación de texto, etc. 
En esta sección vamos a centrarnos en la clasificación de texto, que es una de las áreas más importantes del procesamiento de lenguaje natural que va de la mano con el análisis de sentimientos.

En el mismo sentido, el PLN tiene como objetivo fundamental extraer el significado o la semántica del lenguaje. 
La semántica es el estudio de las palabras y su relación con el mundo real, se refiere la capacidad que se tiene de entender el significado de una oración o un texto. 
No solamente es analizar las palabras y estructuras gramaticales, sino también, de darles un sentido. 
Esto se hace a través de diversas técnicas y algoritmos, en su mayoría basados en el aprendizaje automático y el procesamiento de cantidades masivas de información. 
Todo esto para analizar y extraer patrones, significados y relaciones entre palabras y textos. 
Todo esto permitiendo a las computadoras entender el lenguaje humano, mediante la toma de decisiones de forma autónoma.

En \cite{b6} se menciona el surgimiento del PLN estadístico, este enfoque se basa en la idea de que el lenguaje natural sigue ciertas regularidades estadísticas que se pueden modelar con precisión utilizando técnicas de estadística y aprendizaje automático. Por ejemplo, se pueden utilizar modelos probabilísticos para predecir la probabilidad de que una palabra aparezca en un contexto determinado o para determinar la relación semántica entre palabras.

%Frecuencia y cercanía de las palabras

% b6 Nadkarni, P.M., Ohno-Machado, L. and Chapman, W.W., 2011. Natural language processing: an introduction. Journal of the American Medical Informatics Association, 18(5), pp.544-551.

%
%

\section{Casos de uso o ejemplos}

% En esta sección menciona todos los ejemplos que pueda encontrar, no entre en mucho detalle con algoritmos. La idea es que mencione ejemplos de aplicaciones de análisis de sentimientos.

% La idea es que mencione ejemplos que luego Fredd pueda señalar, mencionar y posteriormente explicar. Es importante que deje en un comentario (en código), las técnicas mencionadas en dichos ejemplos.

% Es importante que sea lo más específico posible, con el objetivo de que se puedan mencionar todos los posibles casos de uso e inflar el contenido.

% Ejemplo:
% Técnicas utilizadas: n-gramas, bolsas de palabras, NPL, etc.
En el paper \cite{b2} se menciona como las empresas X, Y hacen uso de comentarios en línea para determinar su reputación mediante datos obtenidos en la web. Explica como manipulan dichos datos para tomar decisiones...

La investigación \cite{b4} fue hecha para determinar como campañas políticas utilizan tweets para influir en las elecciones. Se menciona como las campañas políticas utilizan hashtags para influir en las elecciones de forma que...  


\section{Algoritmos o técnicas utilizadas}

% En esta sección es en donde más tendrá imágenes que puede utilizar de los mismos papers, hay un 'ejemplo de figura' en el código que puede utilizar para insertar imágenes, se encuentra antes de la bibliografía.

% El enfoque principal de la sección, es que agarre los ejemplos que se mencionaron en el apartado anterior, Randall deberá dejar en un comentario (en código), las técnicas mencionadas en dichos ejemplos. De esta forma, usted podrá investigar y referenciar papers que utilicen dichas técnicas.

\section{Conclusiones}

\subsection{Figures and Tables}
\paragraph{Positioning Figures and Tables} Place figures and tables at the top and 
bottom of columns. Avoid placing them in the middle of columns. Large 
figures and tables may span across both columns. Figure captions should be 
below the figures; table heads should appear above the tables. Insert 
figures and tables after they are cited in the text. Use the abbreviation 
``Fig.~\ref{fig}'', even at the beginning of a sentence.

% Ejemplo de tabla
\begin{table}[htbp]
\caption{Table Type Styles}
\begin{center}
\begin{tabular}{|c|c|c|c|}
\hline
\textbf{Table}&\multicolumn{3}{|c|}{\textbf{Table Column Head}} \\
\cline{2-4} 
\textbf{Head} & \textbf{\textit{Table column subhead}}& \textbf{\textit{Subhead}}& \textbf{\textit{Subhead}} \\
\hline
copy& More table copy$^{\mathrm{a}}$& &  \\
\hline
\multicolumn{4}{l}{$^{\mathrm{a}}$Sample of a Table footnote.}
\end{tabular}
\label{tab1}
\end{center}
\end{table}

% Ejemplo de figura
\begin{figure}[htbp]
\centerline{\includegraphics{fig1.png}}
\caption{Example of a figure caption.}
\label{fig}
\end{figure}

\begin{thebibliography}{00}
\bibitem{b1} Yue, L., Chen, W., Li, X. et al. A survey of sentiment analysis in social media. Knowl Inf Syst 60, 617–663 (2019).
\bibitem{b2} Nandwani, P., Verma, R. A review on sentiment analysis and emotion detection from text. Soc. Netw. Anal. Min. 11, 81 (2021).
\bibitem{b3} PS, S. and Mahalakshmi, G., 2017. Emotion models: a review. International Journal of Control Theory and Applications, 10(8), pp.651-657.
\bibitem{b4} Yaghoobian, H., Arabnia, H.R. and Rasheed, K., 2021. Sarcasm detection: A comparative study. arXiv preprint arXiv:2107.02276.
\bibitem{b5} Gonçalves, P., Araújo, M., Benevenuto, F., and Cha, M. 2013. Comparing and Combining Sentiment Analysis Methods. In Proceedings of the First ACM Conference on Online Social Networks (pp. 27–38). Association for Computing Machinery.
\bibitem{b6} Nadkarni, P.M., Ohno-Machado, L. and Chapman, W.W., 2011. Natural language processing: an introduction. Journal of the American Medical Informatics Association, 18(5), pp.544-551.
\bibitem{b7} Jurafsky, D. and Martin, J.H., 2008. Speech and language processing. 2nd edn Englewood Cliffs. NJ: Prentice Hall.
\end{thebibliography}
\vspace{12pt}
\color{red}
IEEE conference templates contain guidance text for composing and formatting conference papers. Please ensure that all template text is removed from your conference paper prior to submission to the conference. Failure to remove the template text from your paper may result in your paper not being published.

\end{document}
