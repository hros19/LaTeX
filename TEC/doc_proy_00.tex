
\author{Hansol Antay Rostrán}

%%% -------------------------------------
%%% DEFINICIÓN DE COMANDOS (CONTENIDO TEXTUAL)

% Nombre del curso
\newcommand{\nombrecurso}{\textbf{CS3401 - Seminario de estudios filosóficos históricos}}

% Institución
\newcommand{\institucion}{\textbf{Instituto Tecnológico de Costa Rica}}

% Sede
\newcommand{\sede}{\textbf{Centro Académico de Limón}}

% Nombre del profesor
\newcommand{\nombreprofesor}{\textbf{Prof. Mag. Mauricio Cedeño Camacho}}

% Titulo de la asignacion
\newcommand{\tituloasignacion}{\textbf{Impacto ético y moral de la Inteligencia Artificial en un mundo automatizado}}

% Objetivo de la tarea
\newcommand{\objetivo}[1]{
  \fontsize{#1}{#1}\selectfont
  \lipsum[1]
}

% Fecha y hora de entrega
\newcommand{\fechaentrega}{10 de marzo del 2023, a las 9:30 AM}

%%% -------------------------------------
% PARTES DEL DOCUMENTO

%%% TITULO (sede, curso, nombreTarea, profesor)

\newcommand{\titulo} {
  %% Titulo del curso
  \noindent
  \fontsize{14}{14}\selectfont
  \color[rgb]{0.0, 0.0, 0.5}
  \textbf{\sede} \\
  \textbf{\nombrecurso} \\
  \vspace*{0.5cm}
  \fontsize{12}{12}\selectfont
  \color[rgb]{0.0, 0.0, 0.0}
  \textit{\textbf{\nombreprofesor}}
  \\ [2\baselineskip]
  \fontsize{14}{14}\selectfont
  \color[rgb]{0.0, 0.0, 0.5}
  \vspace*{0.25cm}
  \textbf{\tituloasignacion}
  \\
}

%%% INTEGRANTES

\newcommand{\integrantes}{
  \begin{tcolorbox}[colback=yellow!10!white, colframe=black]
    \begin{center}
      \fontsize{12}{12}\selectfont
      \textbf{Integrantes del equipo:}
      \\ \vspace*{0.25cm}
      Hansol Antay Rostrán \\
      Randall Corella Castillo \\
      Duan Espinoza Olívares \\
      Bayron Rodríguez Centeno \\
      Duvan Wing Morales \\
    \end{center} 
  \end{tcolorbox}
}

%%% PORTADA

\newcommand{\portada}{
  \begin{center}
    \vspace*{3.15cm}
    %% LOGO TEC
    \includegraphics[width=0.62\textwidth]{img/itcr logo.png}
    \vspace*{1.2cm}

    %%% Curso, profesor
    \titulo
  
    \vspace*{1.5cm}
    
    %%% Integrantes
    \integrantes

    % \vspace*{1cm}
    
    \vspace*{2cm}
    \fontsize{12}{12}\selectfont
    \color[rgb]{0.0, 0.0, 0.7}
    \textbf{Fecha de entrega:} \fechaentrega
    
    % askaskadsk

  \end{center}
}

%%% ------------------------------------
%%% Inicio del documento

% The document class is report
\documentclass[11pt]{report}

%%% PAQUETES
\usepackage{xcolor}
\usepackage{geometry}
\usepackage{ragged2e}
\usepackage{graphicx}
\usepackage{lipsum}
\usepackage{tcolorbox}
\usepackage[spanish]{babel}
\usepackage[T1]{fontenc}
%ToC
\usepackage{tocloft}

% Margén del documento
\geometry{left=2.5cm,right=2.5cm,top=2.5cm,bottom=2.5cm}

% Texto de la tabla de contenidos
\addto\captionsspanish{% Replace "english" with the language you use
  \renewcommand{\contentsname}%
    {Índice}
}

% Hacer que las secciones empiecen con 1.1, 1.2, etc.
\renewcommand{\thesection}{\arabic{section}}

% Hacer que las subsecciones empiecen con 1.1.1, 1.1.2, etc.
\renewcommand{\thesubsection}{\thesection.\arabic{subsection}}

% Cambiar el espacio entre las filas de la tabla de contenidos
\setlength{\cftbeforesecskip}{0.5em}

% Cambiar la indentación entre la numeración y el texto de la tabla de contenidos
\cftsetindents{section}{0cm}{0.4cm}
\cftsetindents{subsection}{0.25cm}{0.8cm}
\cftsetindents{subsubsection}{0.5cm}{1.2cm}

% Poner las secciones en negrita
\renewcommand{\cftsecfont}{\bfseries}


% Quitando la numeración de las secciones y subsecciones del texto

% Poner las subsubsecciones en cursiva
\renewcommand{\cftsubsubsecfont}{\itshape}


% Idioma del documento
\usepackage[spanish]{babel}

\begin{document}
\renewcommand\cftchapdotsep{\cftdotsep}
\renewcommand\cftchapleader{\cftdotfill{\cftchapdotsep}}

  \thispagestyle{empty}
  
  \portada % Lineas 72-113

  \newpage
  
  \tableofcontents

  \newpage

  %\section{Introducción} % Extra

  % ------------------------------------------------
  \section{Objetivos}
  

  \subsection{Objetivo general}
  El objetivo general de esta investigación es evaluar el impacto ético y moral de la Inteligencia
  Artificial en un mundo automatizado, con el fin de proporcionar una guía para la toma de
  decisiones sobre el diseño, desarrollo e implementación de este tipo de tecnologías.

  \subsection{Objetivos específicos}
  \begin{itemize}
    \item Evaluar y denota las implicaciones que tendría la Inteligencia Artificial en la violación
    de la privacidad y seguridad de los datos personales de las personas.
    \item Proporcionar una guía para la toma de decisiones sobre el diseño, desarrollo e
    implementación de sistemas basados en Inteligencia Artificial, de forma que se minimicen los
    riesgos éticos y morales.
    \item Analizar riesgos actuales y potenciales asociados al uso de la Inteligencia Artificial y
    como estos pueden afectar a la sociedad en el ámbito económico, laboral y ético-social.
  \end{itemize}

  % ------------------------------------------------
  \section{Justificación}
  La Inteligencia Artificial cada día se vuelve más avanzada y ya está siendo utilizada en una gran
  variedad de sectores, como la medicina, la educación, la industria, la seguridad, etc. Sin embargo,
  es importante evaluar el impacto que esta tecnología tiene en la sociedad y en las personas desde
  el punto de vista ético y moral.

  Siempre ha existido la preocupación de que tanto impacto puede tener la Inteligencia Artificial en
  la era de la automatización. Esta investigación tiene como objetivo proporcionar una visión general
  y guía para la toma de decisiones sobre el impacto ético y moral de la Inteligencia Artificial
  mediante el diseño, implementación y su aplicación en la industria.

  Un enfoque muy claro de este estudio será el señalar los riesgos potenciales de la Inteligencia
  Artificial, sobre todo en el ámbito de privacidad y seguridad de los datos, y cómo estos pueden
  afectar a las personas. Además, se analizarán y propondrán soluciones para estos problemas.

  % ------------------------------------------------
  
  % \section{Marco teórico}
  % Entrega 2

  % ------------------------------------------------
  %\section{Metodología}
  % Extra

  %\section{Resultados}

  % ------------------------------------------------
  %\section{Conclusiones y recomendaciones}
  % Entrega 3
  
\end{document}
