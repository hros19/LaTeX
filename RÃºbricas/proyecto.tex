
\author{Hansol Antay Rostrán}

%% Definición de comandos

\newcommand{\institucion}{Instituto Tecnológico de Costa Rica}
\newcommand{\nombrecurso}{Fundamentos de Organización de Computadoras}
\newcommand{\nombreproyecto}{Proyecto 01 - Simón Dice}
\newcommand{\valorporcentual}{15}
\newcommand{\anolectivo}{
    \year
}
\newcommand{\nota}{14}

\newcommand{\notamaxima}{15}

\newcommand{\notareprobado}{10} % si es menor a esto, sale rojo

\newcommand{\porcentajeobtenido}{
    \numexpr\nota/\notamaxima*\valorporcentual\relax
}
\newcommand{\estudiantes}{
    Estudiante 1, Estudiante 2, Estudiante 3
}

\documentclass[11pt]{article}

%%% PAQUETES
\usepackage{xcolor} % Para definir colores
\usepackage{geometry} % Para definir el margen del documento
\usepackage{ragged2e} % Para justificar el texto
\usepackage{graphicx} % Para incluir imágenes
\usepackage{lipsum} % Para generar texto de prueba
\usepackage{tcolorbox} % Para los cuadros de texto
\usepackage[spanish]{babel} % Idioma
\usepackage[T1]{fontenc} % Fuente (opcional)
\usepackage{tocloft} % Para cambiar la tabla de contenidos
\usepackage{tikz} % Para dibujar figuras
\usepackage{titleps} % Para el encabezado
\usepackage{sectsty} % Para cambiar el formato de las secciones
\usepackage{hyperref} % Para agregar hipervínculos
\usepackage{lastpage} % Para mostrar el número de página
\usepackage{titletoc}
\usepackage{colortbl} % Para cambiar el color de las celdas de la tabla
\usepackage{float}
\usepackage{datetime}
\usepackage[spanish]{babel}
\usepackage[T1]{fontenc}
\usepackage{tgadventor}
\usepackage{amsmath}
\usepackage{pgfmath}
\usepackage{siunitx}
\usepackage{tikz}

\renewcommand{\familydefault}{\sfdefault}

\geometry{left=2.5cm,right=2.5cm,top=2.5cm,bottom=2.5cm}

% Definición de los colores de las secciones y subsecciones
\definecolor{darkblue}{rgb}{0,0,0.5}
\definecolor{lightblue}{rgb}{0.2, 0.4, 0.8}
\definecolor{darkorange}{rgb}{0.9, 0.4, 0}
\definecolor{lightred}{rgb}{0.9, 0.2, 0.2}
\definecolor{verylightyellow}{rgb}{1, 1, 0.8}

\begin{document}
    \noindent
    % No borrar
    % Cambiar el tamaño de la fuente
    \fontsize{12}{14}\selectfont
    % Cambiar el color de la fuente
    \color{darkblue}
    \textbf{\institucion} \\
    \textbf{\nombrecurso} \\
    % Cambiar el tamaño de la fuente
    \fontsize{10}{12}\selectfont
    % Cambiar el color de la fuente
    \color{black}
    \newline
    \textbf{Asignación: } \nombreproyecto \\
    \textbf{Valor: } \valorporcentual\% \\
    \textbf{Año lectivo: } {
        % comando para mostrar el año actual
        \the\year
    } \\
    \textbf{Semestre:} {
        % Si el mes es menor a 7, entonces es Primer Semestre, de lo contrario es Segundo Semestre
        \ifnum\month<7 Primer Semestre \else Segundo Semestre \fi
    } \\

    \newcommand{\colornota}{
        % si es menor a 70, entonces retorna red, de lo contrario retorna green
        \ifnum\nota<\notareprobado red \else green \fi
    }

    \definecolor{golden}{rgb}{1, 0.84, 0}
    \definecolor{softred}{rgb}{1, 0.4, 0.4}
    
    \noindent
    \begin{tikzpicture}
        \node[rectangle, fill=black, text=white, text width=19cm*0.7, minimum height=1cm] at (0,0) {\estudiantes};
        % if \nota is less than 70, create a red rectangle, if is 100, create a golden rectangle, otherwise create a green rectangle

        % Use default LaTeX font
        \fontfamily{cmr}
        \ifnum\nota<\notareprobado
            \node[rectangle, fill=softred, text=black, text width=9.5cm*0.3, minimum height=1cm] at (12cm*0.7,0) {
                \textbf{$\frac{\nota}{\notamaxima}$} \ pts.
                    % Redondear lo de abajo a 2 decimales
                    \pgfmathparse{\nota / \notamaxima * 100}
                    % Set sfdefault as font for the next text
                    (\num[round-mode=places,round-precision=1]{\pgfmathresult}\%)
            };
        \else
            \ifnum\nota=\notamaxima
                \node[rectangle, fill=golden, text=black, text width=9.5cm*0.3, minimum height=1cm] at (12cm*0.7,0) {
                    \textbf{$\frac{\nota}{\notamaxima}$} \ pts.
                    % Redondear lo de abajo a 2 decimales
                    \pgfmathparse{\nota / \notamaxima * 100}
                    % Set sfdefault as font for the next text
                    (\num[round-mode=places,round-precision=1]{\pgfmathresult}\%)
                };
            \else
                \node[rectangle, fill=green, text=black, text width=9.5cm*0.3, minimum height=1cm] at (12cm*0.7,0) {
                    \textbf{$\frac{\nota}{\notamaxima}$} \ pts.
                    % Redondear lo de abajo a 2 decimales
                    \pgfmathparse{\nota / \notamaxima * 100}
                    % Set sfdefault as font for the next text
                    (\num[round-mode=places,round-precision=1]{\pgfmathresult}\%)
                };
            \fi
        \fi

        \fontfamily{sfdefault}
    \end{tikzpicture}
\end{document}